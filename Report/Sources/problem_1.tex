% !TeX spellcheck = en_US
\section{Problem 1}
In this exercise we need to find the minimum of the given 2-dimensional function:\\
\begin{equation}	
		F(\mathbf{w})=w_{1}^{2}+w_{2}^{2}+(0.5w_{1}+w_{2})^{2}+(0.5w_{1}+w_{2})^{4}
	\label{eq:function1}	
\end{equation}
with the Conjugate Gradient (Fletcher-Reeves) method and the Gradient Descent.\\

Initially, we can conclude that the function $F(w)$ is not in quadratic form because of the term $(0.5w_{1}+w_{2})^{4}$.
A function is said to be in quadratic form if it can be expressed as a second-degree polynomial where all the terms are either squared terms or cross-products of the variables. The presence of the fourth-degree term $(0.5w_{1}+w_{2})^{4}$.
makes this function a higher-degree polynomial, specifically a quartic function with respect to $(0.5w_{1}+w_{2})$, which means it cannot be classified as quadratic.\\
Also, the independent values in this function are $w_{1},w_{2}$, because only with them we can manipulate the $F(w)$.\\ 

As an initial guess we have $w\left(0\right) = \left[3, 3\right]^T$.\\
Τhe steps we have to use are specific for each iteration

\begin{center}
	\underline{\textit{FIRST ITERATION k = 0}}
\end{center}

\underline{Step1: Calculate the Gradient at  $w\left(k\right)$ }\\
\(\nabla f(w_1,w_2) = \left(\begin{array}{c}
	\dfrac{\partial f}{\partial w_1} \\[4mm]
	\dfrac{\partial f}{\partial w_2}
\end{array}\right)\) = $\left(\begin{array}{c}
	2w_1 + (0.5w_1+w_2) + 2(0.5w_1+w_2)^3\\[1mm]
	2w_2 + 2(0.5w_1+w_2) + 4(0.5w_1+w_2)^3
\end{array}\right) = \left(\begin{array}{c}
	2.5w_1 + w_2 + 2(0.5w_1+w_2)^3\\[1mm]
	w_1 + 4w_2 + 4(0.5w_1+w_2)^3
\end{array}\right)$ \\[3mm]

where at the point $w\left(0\right) = \left[3, 3\right]^T$ we have $\nabla f(x) = \left(\begin{array}{c}
	-53 \\
	-19
\end{array}\right)$
\\[4mm]

\begin{center}
	\underline{\textbf{GRADIENT DESCENT}}
\end{center}
Gradient descent is one of the most favored optimization technique, because of its simplicity and its generality.\\
Given the same initial guess $w\left(0\right) = \left[3, 3\right]^T$. \\
In order to find the minimum of the function~\ref{eq:function1} we need to follow the Gradient Descent's steps:\\

\underline{Step 1: Initialize initial point $w_{0}$, value of tolerance $tol$ and Step size}\\
As we have mentioned, the initial values are:
\begin{itemize}
	\item $w\left(0\right) = \left[3, 3\right]^T$
	\item $tol = 10^{-6}$
	\item $\alpha = 0.01$\\
	The exercise says that we apply the GD method with unit step movement. However, if we set $\alpha = 1$, which is considered a large value, the algorithm may not converge to the optimal point and it will even diverge completely. That's why we set $\alpha = 0.01$. 
\end{itemize} 
\vspace{2mm}

\underline{Step 2: Compute gradient $\nabla f$ at $w_{k}$}
\vspace{4mm}

\underline{Step 3: Make a scaled step in the opposite direction to the gradient}\\
\begin{equation}
	\begin{gathered}
		\text{step} = \alpha \cdot \nabla f_{k}\\
	\end{gathered}
\end{equation}
\vspace{2mm}

\underline{Step 4: Check convergence}\\
If the step size is smaller than the convergence, then we stop. Else, we repeat Step 2 and Step 3 while we update $x$.
\vspace{4mm}

\underline{Step 5: Update $x$}
\begin{equation}
	x_{k+1} = x_{k} - \alpha \cdot \nabla f_{k}
\end{equation}
\vspace{4mm}

Taking these steps into consideration we will attempt to minimize our function.\\

\begin{center}
	\underline{\textit{FIRST ITERATION k = 0}}
\end{center}




