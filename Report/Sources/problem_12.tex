% !TeX spellcheck = en_US
\section{Problem 12}

To compute the number of weights and biases for each convolutional layer, we need to take into consideration the size of kernels and the number of input/output channels for each layer.

\subsection{First hidden layer}
For the first hidden layer we have:
\begin{itemize}
	\item \textbf{Input Channels:} 3
	\item \textbf{Kernel Size:} 3
	\item \textbf{Output Channels:} 4
\end{itemize}

Each filter in a convolutional layer has a weight for each entry in the kernel in the kernel for each input channel, and there is one bias per filter.\\ 

The number of weights for a single filter in the first layer is the kernel size multiplied by the number of input channels:
\[
\textit{Weights per filter} = \textit{Kernel size} \times \textit{input channels} = 3 \times 3 = 9
\] \\ 

Since there are $4$ filters, the total number of weights for the first layer is: 
\[
\textit{Total weights} = \textit{Weights per filter} \times \textit{Filters} = 9\times 4 = 36.
\] \\ 

There's only one bias per filter, so the total number of biases is $4$.

\subsection{Second hidden layer}
For the second hidden layer we have:
\begin{itemize}
	\item \textbf{Input Channels:} 4 (\textit{\small from the previous layer})
	\item \textbf{Kernel Size:} 5
	\item \textbf{Output Channels:} 10
\end{itemize}

The number of weights for a single filter in the second layer is the kernel size multiplied by the number of input channels from the first layer:
\[
\textit{Weights per filter} = \textit{Kernel size} \times \textit{Input channels from previous layer} = 5 \times 4=20
\]
Since there are 10 filters, the total number of filters on the second layer is $20 \times 10 = 200$.

As far as the biases are concerned, there's only one bias per filter, so for $10$ filters, the total number is $10$.\\

Summarizing, for the two convolutional layers, we need a total of \underline{$36$ weights and $4$ biases for the first layer} and \underline{$200$ weights and $10$ biases for the second layer}.
\vspace{3mm}